\section{The Problem Statement}
The problem of fault detection can be defined mathematically using a model where the $n^{\rm{th}}$ instance of a band-limited signal $x(n)$,  used as an input is given as 
\begin{equation}
	x(n) = \sum\limits_{i = 1}^{q} {\left| {{A_i}} \right|{e^{j(n{\omega _i[n]} + {\phi _i})}}}  + v(n),
	\label{eq:signalModel}
\end{equation}
where ${\left| {{A_i}} \right|}$, ${\omega _i[n]}$, $\phi_i$ are the individual amplitude, normalized instantaneous frequency, and phase of the $i^{\mathrm{th}}$ component, respectively. ${\omega _i[n]} = 2\pi {f_i[n]}/F_s$, $j = \sqrt { - 1}$, $F_s$ is the sampling rate (samples/s), and $e$ is the exponent operator, respectively. ${f_i[n]}$ is the $i^{\mathrm{th}}$ component's instantaneous frequency (Hz). $v(n)$ is the additive white Gaussian noise with zero mean and variance $\sigma_v ^2$, i.e., $v[n] \sim \mathcal N(0,\sigma _v^2)$. $x(n)$ consists of ${\tilde q}$ number of high-amplitude  dominating frequency components out of $q.$ We assume that the frequency has temporal variation. Any diversion otherwise can be accommodated using a constant frequency, i.e., $\omega_i (n) = \omega_i.$ 

The process of fault detection starts with selecting and recording an appropriate signal that should be convenient to acquire and carries vital features of the fault. The second problem is the pre-conditioning of the input signal to achieve maximum interpretability and fault information. The preprocessing is accomplished by the targeted elimination of ${\tilde q}$ dominant components. The variable frequency drives (VFD) aggravates the problem as the supply frequency can vary with time. After signal conditioning, the third problem is to extract vital features that carry information about the faults. For the model described by (\ref{eq:signalModel}), feature extraction mainly involves the estimation of frequency and amplitude of the remaining $(q-\tilde q)$ components. Lastly, based on the extracted features, fault detection and identification is carried out.