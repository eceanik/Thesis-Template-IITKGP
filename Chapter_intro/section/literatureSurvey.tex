\section{Survey of Literature}
Considering the different problems associated with the detection of SCIM faults, we have divided the review of existing literature into different sections. We start with a description of various faults that affect the SCIM, followed by an account of motor-parameters used for fault detection. The next section reviews existing fault diagnosis technologies. Spectral estimation is an essential part of fault detection. As a result, we devote the following section to review available spectral estimators. The subsequent section examines the IF estimation and signal conditioning literature. 

\subsection{Squirrel Cage Induction Motor Faults}
Various surveys have estimated that mechanical faults of damaged bearings constitute most SCIM faults \cite{Riera-Guasp2015}, followed closely by stator faults. Broken rotor bars (BRB) accounts for about 5-10\% of all induction motor faults \cite{Nandi2005}. Improvements in the fabrication process have reduced the occurrence of rotor faults. Still, existing challenges of rotor fault detection continue to foster relevant research \cite{Nandi2005}. Interestingly, in recent times, the study of stator fault takes precedence over the rotor and bearing defects, as emphasized by Fig. \ref{Fig:faults}.
\begin{figure}[h] \centering
	{\includegraphics[width=100mm]{Chapter_intro/figs/publicationPerFaults.eps}} \caption{Faults undertaken in published works from 2011 to June 2020. Source: Scopus.} \label{Fig:faults}
\end{figure}
A brief overview of different faults that plague the induction motors are described below:
%
\subsubsection{Bearing Faults}
As depicted in Fig. \ref{Fig:bearingDefects}, bearing faults can be classified into three major categories depending upon the damage to a particular component. These faults are outer-raceway fault, inner-raceway fault, and rolling element fault. Apart from these, faults can also plague the cage that holds the rolling elements. 
\begin{figure}[h] \centering
	{\includegraphics[width=150mm]{Chapter_intro/figs/bearingDefects.eps}} \caption{Different types of bearing defects. Inner-raceway fault, outer-raceway fault, and rolling-element fault.} \label{Fig:bearingDefects}
\end{figure}
%
\begin{enumerate}
	\renewcommand{\theenumi}{\roman{enumi}}
	\item \emph{Outer-raceway fault:} Point faults in the outer raceway exhibit characteristic frequency components in the vibration spectrum, according to (\ref{outerRace})
	\begin{equation}
		\label{outerRace} {f_{ov}} = \left( {\frac{N_b}{2}}	\right){f_r}\left[ {1 - \frac{{{b_d}}}{{{d_\rho }}}\cos \beta }
		\right],
	\end{equation}

	\item \emph{Inner-raceway fault:} Similarly, damage to the inner raceway gives fault-specific frequency components in the vibration spectrum as
	\begin{equation}
		\label{innerRace} {f_{irv}} = \left( {\frac{N_b}{2}} \right){f_r}\left[ {1 + \frac{{{b_d}}}{{{d_\rho }}}\cos \beta }
		\right],
	\end{equation}
	\item \emph{Rolling element fault:} caused due to point defects in the rolling element giving rise to specific frequency components in the vibration spectrum as provided by
	\begin{equation}
		\label{ball}{f_{rv}} = \left( {\frac{{{f_r}{d_\rho }}}{{2{b_d}}}} \right)\left[ {1 - {{\left( {\frac{{{b_d}}}{{{d_\rho }}}\cos \beta
					} \right)}^2}} \right],
	\end{equation}
	\item \emph{Cage fault:} the rolling elements are held together in their place by an enclosed cage-like structure. Any damage to the cage gives specific a frequency component in the vibration as  
	\begin{equation}
		\label{cage} 
		{f_{cv}} = \left( {\frac{{{f_r}}}{2}} \right)\left[ {1 - {{\left( {\frac{{{b_d}}}{{{d_\rho }}}\cos \beta } \right)}^2}} \right],
	\end{equation}
\end{enumerate} 
%
where ${f_{ov}}$, ${f_{irv}}$, ${f_{rv}}$, and ${f_{cv}}$ are the characteristic vibration fault frequency components for the outer raceway, inner raceway, rolling element, cage, respectively. $f_r$ is the rotational frequency. $N_b$, $b_d$, $d_\rho$, and $\beta$ are the number of balls, ball diameter, ball pitch diameter, and the contact angle between the ball and the races, respectively. The theoretical fault signatures using stator current can be obtained from the corresponding vibration signatures as follows
%
\begin{equation}
	\label{bearingCurrentSignature}
	f_{xi} = f_o \pm mf_{xv},
\end{equation}
where $f_{xi}$ and $f_{xv}$ are the theoretical current and voltage signature of $x$-type bearing fault condition and $m$ is any integer.
\subsubsection{Stator faults}
Stator windings are the most operation critical and fault-prone elements of a SCIM. The main reason for stator winding faults can be attributed to the winding's deteriorating insulation with time. The failure modes associated with winding faults are short-circuit, open-circuit, stator-core faults \cite{drif2014stator}. The insulation degradation causes turn-to-turn defects, and the corresponding stator-current spectral signature is given by
\begin{equation}
	\label{itsc1}
	{f_{si}} = {f_o}\left[ {\frac{{m\left( {1 - s} \right)}}{{2p}} \pm k} \right],
\end{equation}
where $m=0,1,2,\ldots$ and $k=0,1,2,\ldots$ The inherent ambiguity of (\ref{itsc1}) and presence of the same signature in case of eccentricity faults limit its use. As an alternative, the following signature has been used to detect the turn-to-turn short-circuit faults \cite{wolkiewicz2016online}
\begin{equation}
	\label{itsc2}
	{f_{si}} = {f_o}\left[ {\frac{{m{N_r}\left( {1 - s} \right)}}{{2p}} \pm k} \right],
\end{equation}
where $N_r$ are the number of rotor bars, $p$ is the number of poles of the motor, $m=1,2,3,\ldots$ and $k=1,3,5,\ldots.$
\subsubsection{Broken Rotor Bar fault}
Continual running of the motors with unbalanced magnetic pull and inherent eccentricity leads to BRB with small cracks at the junction between the end-ring and the bars. The signatures for broken and cracked rotor bars on the current spectrum are due to the rotor circuit asymmetries \cite{Henao2014, kia2005, didier2006}, which give rise to multiple frequency components around the fundamental in the stator current spectrum as
\begin{equation}
	\label{fBRB}
	{f_{bi}(\pm k)} = (1 \pm 2ks)f_o
\end{equation}
where $k$ is any integer, $s$ is the slip, and $f_o$ is the mains supply frequency. The BRB signature in the vibration spectrum is given as
\begin{equation}
	{f_{bv}(\pm k)} = {f_r} \pm k({f_s} - {f_r})p,
	\label{brbSig}
\end{equation}
where $f_s=N_s/60$ is the synchronous frequency, with the synchronous speed $N_s$ in rpm.
\subsubsection{Eccentricity Faults}
Eccentricity faults occur when an uneven air-gap exists between the stator and the rotor \cite{Nandi2005}. Depending on the non-uniform air-gap characteristics, the eccentricity can be classified into different categories, viz; static, dynamic, and mixed conditions.  An eccentric rotor can lead to an unbalanced magnetic pull, leading to BRB and stator-rotor rub. The unbalanced magnetic field induces fault-specific frequency components in the motor armature current given by \cite{Nandi2005,Gyftakis2013}
\begin{equation}
	\label{intro_Ecc}
	{f_{ei}} = \left[ {\left( {kR \pm {n_d}} \right)\frac{{\left( {1 - s} \right)}}{p} \pm v} \right]{f_o}
\end{equation}
Where $R$ is the number of rotor slots, ($v = \pm 1, \pm 3, ...$) are the stator time-harmonic order present in the power supply driving the motor, and $n_d$ is the variable depending on which eccentricity related faults can be classified as given below:
\begin{enumerate}
	\renewcommand{\theenumi}{\roman{enumi}}
	\item \emph{Static Eccentricity} - The axis of rotation does not coincide with the stator axis but is the same as the rotor axis. As a result, a non-uniform stationary air-gap is created, which does not rotate with the rotor. Static eccentricity is caused due to ovality of the stator or poor positioning of the rotor. Mathematically $n_d = 0$ in (\ref{intro_Ecc}). Static eccentricity fault is illustrated in Fig. \ref{Fig:introEcc}(a). 
	%
	\item \emph{Dynamic Eccentricity} - With dynamic eccentricity, the  rotor-axis and the rotational axis do not coincide. However, the rotational axis and the stator axis are the same, and the position of the minimum air gap rotates with the rotor. Mathematically, frequency components can be modeled by putting $n_d = 1,2,3,...$ in (\ref{intro_Ecc}). This phenomenon can be visualized in Fig. \ref{Fig:introEcc}(b). This misalignment is caused by several factors such as bearing wear, bent rotor shaft, misalignment, and mechanical resonance due to shaft speed oscillation.
	%
	\item \emph{Mixed Eccentricity} - In practice, both the static and dynamic eccentricities exist together. All the motors studied in this thesis have confirmed this fact. The static and dynamic eccentricity coexists, leading to the mixed eccentricity condition. In mixed eccentricity, none of the three centers coincides as illustrated in Fig. \ref{Fig:introEcc}(c).
\end{enumerate}
%
\begin{figure}[h]
	%\ContinuedFloat
	\centering \mbox{
		\subfloat[{\scriptsize Static eccentricity}]{\includegraphics[width=50mm]{Chapter_intro/figs/static_ecc.eps}}
		\subfloat[{\scriptsize Dynamic eccentricity}]{\includegraphics[width=50mm]{Chapter_intro/figs/dynamic_ecc.eps}}
		\subfloat[{\scriptsize Mixed eccentricity}]{\includegraphics[width=50mm]{Chapter_intro/figs/mix_ecc.eps}}}
	\caption{Illustrations of different eccentricity faults}
	\label{Fig:introEcc}
\end{figure}
The mixed eccentricity condition exists inherently in motors. As a result, most literature uses this mixed eccentricity component to detect the level of eccentricity present. The mixed eccentricity spectral signature in current is defined by (\ref{fMixedCurrent}). This component has been used in this thesis for speed estimation.
\begin{equation}
	\label{fMixedCurrent}
	{f_{mi}} = \left| {\frac{{1 \pm k(1 - s)}}{p}} \right|{f_o}
\end{equation}
For the vibration signal, the mixed eccentricity component is given by \cite{Nandi2005}
\begin{equation}
	\label{mixedEccVibration}
	{f_{mv}} = 2f_o \pm f_r.
\end{equation}

This thesis mainly analyzed stator current and motor vibration to detect faults in SCIM. However, multiple other parameters that are analyzed to assess the fault condition of a motor are given below: 
\subsection{Choice of Signals for Fault Detection}
%
Multiple signals have been used for condition monitoring of SCIMs. A survey of the keywords from recent publications found stator current to be the most widely used signal, followed by vibration, as shown in Fig. \ref{Fig:inputSignal}.
\begin{figure}[h] \centering
	{\includegraphics[width=100mm]{Chapter_intro/figs/inputSignal.eps}} \caption{Input parameters used for detecting faults in published works from 2011 to June 2020. Source: Scopus} \label{Fig:inputSignal}
\end{figure}
A brief review of signals that are used as input for fault detection is given below:
\begin{enumerate}
	\renewcommand{\theenumi}{\roman{enumi}}
		\item \emph{Current} is most popular due to its ease of acquisition and low-cost \cite{garcia2019approach, pandarakone2018evaluating}. Conventionally known as motor current signature analysis (MCSA), it can detect most of the SCIM faults.  MCSA requires only nameplate motor parameters, and the use of clamp-type Hall sensors makes the acquisition non-invasive. The sensors can be clamped around the supply-lines, and their proximity from the motors is inconsequential. However, fault detection using current requires elegant signal processing algorithms to detect weak faults \cite{Zhou2009}, especially for motors operated by VFD due to the presence of fault-imitating harmonics \cite{Soualhi2013}. 
%		
		\item \emph{Vibration signals} effectively detect early-stage mechanical faults and have lower spurious harmonic content \cite{corne2015comparing}. Vibration has been recently used in \cite{shao2018highly, xie2019electromagnetic, contreras2019quaternion, pan2016fault}. The use of vibration signals for VFD-motor can be found in \cite{delgado2015comparative}. In \cite{naha2017mobile}, we have used the inbuilt accelerometer of a smartphone to measure vibration for fault detection. One of the significant drawbacks is the effective placement of sensors. Additionally, it requires an elaborate instrumentation arrangement for proper working and is very costly and fragile.
%		
		\item \emph{Magnetic fields} for fault detection of SCIMs deals with the acquisition of stray flux \cite{park2018stray}, air-gap flux \cite{soleimani2018broken}, and radial flux \cite{surya2016simplified}. Acquisition of flux requires a complex sensor arrangement and fixations for an individual motor. Search coil-based methods require installing pick-up coils near stator slots and the motor's frame to capture the flux pattern. However, inaccessible motors and installation of the sensor on existing systems render the use of search coils infeasible. It is also challenging to locate search coils' best position to capture the signal modulated by the faults efficiently. Moreover, static and dynamic characteristics of the search coil might interfere with the detection procedure. 
%		
		\item \emph{Supply voltage modulation} is also used for the detection of SCIM faults \cite{Nemec2010}. However, the use of voltage requires the sensor to be attached to either of the supply or motor terminals. Also, the acquisition of high-voltage signals requires sophisticated data acquisition, which can be costly and unsafe.
%		
		\item \emph{Active-reactive power analysis} \cite{Cruz2012, Drif2012} was popularized due to its effectiveness for analyzing the motor under time-varying loads and its ability to distinguish between rotor fault and load torque oscillations. However, acquiring voltage signals for this method is challenging.
%		
		\item \emph{Acoustic signals} were employed by \cite{glowacz2019fault, hemamalini2018rational} to detect faults in SCIMs. Though potent for a single motor, the problem escalates when multiple systems operate in tandem. Detection and isolation of faults from numerous sources within an enclosed environment are arduous.
%
		\item \emph{Thermal field analysis} \cite{Ying2010}, \emph{thermal imaging} \cite{osornio2018recent}, and \emph{temperature} \cite{mohammed2018stator}, although used for fault detection in various heavy industries, are not very popular for SCIMs. Extracting fault-significant information from thermal images is complicated due to the metal enclosures. Although, It can be used for detection of the stator and the bearing related failures.
\end{enumerate}

Signals like flux \cite{Cabanas2011} and instantaneous power factor with phase \cite{Drif2012} were used to alleviate the problems faced by MCSA due to load-torque. However, the acquisition of flux and voltage for power factor are major hurdles. As a result, BRB indicators, independent of load-torque oscillations, were proposed \cite{Bruzzese2008, Kim2015, Yang2014}. Few other signals used for SCIM fault detection are rotational speed \cite{Hamadache2015}, efficiency \cite{garcia2018efficiency}, slot harmonics \cite{Khezzar2009}, torque \cite{Gyftakis2013}, and low-voltage offline testing \cite{Kang2015}. 

In \cite{stief2019pca}, data from multiple sensors like vibration, current, acoustic, and voltage have been aggregated using principal components with posterior estimates. The quaternion coefficients of the stator current and signals from a tri-axial vibration sensor have been used for a decision tree-based classification in \cite{contreras2019quaternion}. The use of Sugeno fuzzy integral-based fusion of current and vibration signals is reported in \cite{liu2019fusion}. The use of data from multiple combined sensors reduces the classification errors of a fault detector. However, using multiple expensive sensors for a single motor turns out to be unsuitable for large-scale implementation.

With the logged motor parameter, we need to select an appropriate method for detecting the faults. A review of the recent fault diagnostic procedures are described below:  

\subsection{Review of SCIM Fault Detection and Classification Algorithms}
A condition monitoring system for early detection of SCIM faults can significantly enhance any industry's operation efficiency. Different methods have been proposed in the literature for detecting SCIM faults, as shown in Fig. \ref{Fig:methods}.
\begin{figure}[h] \centering
	{\includegraphics[width=120mm]{Chapter_intro/figs/methods.eps}} \caption{Broad classification of methods for detecting faults in published works from 2011 to June 2020. Source: Scopus.} 
	\label{Fig:methods}
\end{figure}

The central premise of fault diagnosis is concerned with the estimation of fault-frequency components and their amplitudes. Spectral estimation using fast Fourier transform (FFT) \cite{xie2019electromagnetic} and over its envelope \cite{rahman2017online} has been very accurate in detecting different faults in SCIMs. High-resolution spectral estimators like MUSIC \cite{Naha2016, Garcia-Perez2011} and ESPRIT \cite{Kim2013, Xu2012a} have gained prominence over the classical power spectrum because of their robustness and resolution capacity for detecting faults under low load conditions. Still, there are concerns over critical issues related to computational complexity and accurate amplitude estimation of the detected fault components. Parametric spectral estimators for fault diagnosis using maximum likelihood estimation (MLE) with a given signal model can be found in \cite{choqueuse2015induction}. Spectral analysis based on total least square methods was developed by \cite{bouleux2013oblique} for fault detection.

Time-frequency analysis for variable-load conditions using wavelet packet decomposition \cite{Teotrakool2009}, complex-wavelets \cite{Seshadrinath2014}, and tunable Q-factor wavelets \cite{he2015automatic} are widespread. The decomposition of the vibration signal into its intrinsic modes using EMD and subsequent classification of faults by a neural network can be found in \cite{saucedo2016multifault}. The use of complete ensemble EMD to avoid mode mixing has been demonstrated in \cite{delgado2017methodology}. Detection of faults under non-stationarity conditions has been shown with variational mode decomposition in \cite{yan2019application}. Further investigations for concentrating the time-frequency energy using phase information with high-order synchro-squeezing is conducted in \cite{tu2019demodulated}. The use of over-complete dictionaries to represent a signal in the sparse domain and retrieving the frequencies using orthogonal matching pursuit is shown in \cite{morales2018incipient}. A sparsity-based method with group lasso is presented in \cite{zhao2019enhanced}. In \cite{naha2017low}, we have used the sub-Nyquist strategy to design a low-complexity fault detection algorithm. The use of entropy and mutual information as features and support vector machine (SVM) as a classifier using vibration can be found in \cite{pan2016fault}. 

Alternatively, \cite{dalvand2018detection} have used linear prediction for recovering fault-specific signals from the noise to detect point faults and general roughness of bearings. Advances in machine learning (ML) tools in conjunction with stationary wavelet extraction, followed by a support vector machine with an artificial immune system for classification was shown in \cite{abid2018distinct}. The remaining useful life of bearings with empirical Bayes was carried out by extracting the features with complete ensemble EMD \cite{wu2019degradation}. Robust thresholding with historical data of a motor was demonstrated in \cite{schmidt2019open}. The effect of simultaneous occurrence of static eccentricity, BRB, and speed ripples was studied analytically and experimentally in \cite{Kaikaa2014}. A winding function model-based method with parameter estimation using current, rotor speed, and torque are presented in \cite{Shi2014new}. The fault severity is indicated by the magnitude of the fault component \cite{Bellini2001, Xu2010}. However, subspace-based methods cannot give exact information about the amplitude of the fault components. Hence, simulated annealing algorithm was used to determine the correct amplitude and eventual fault severity \cite{Xu2012a}. In \cite{Kim2013, Trachi2016}, the amplitude was estimated using least square estimation, equivalent to computing the discrete Fourier transform (DFT) for a single frequency. However, DFT is not suitable for estimating closely spaced sinusoids \cite{Halder1997, Stoica2000}. Moreover, these methods require extra computational resources for their execution when used in conjunction with MUSIC and ESPRIT.

Modulation of the stator current due to partial BRB is weak. Its detection in a light load condition is challenging as the fault components are very close to the fundamental \cite{Xu2010} and have low amplitude \cite{Concari2008}. Detecting weak faults require the fundamental frequency to be suppressed effectively without affecting the closely spaced fault components. Commonly, this is achieved by using a sharp notch filter \cite{ayhan2008}. But, variable frequency operation with load changes requires the notch filter's central frequency to track the fundamental frequency and its cut-off bandwidth to be adaptive to the slip. In developing unsupervised fault detectors, implementing a notch filter with these characteristics is inconvenient. Therefore, an extended Kalman filter-based signal conditioning method is adopted to remove the fundamental component. This method tracks and attenuates only the fundamental component to improve the detection of close sidebands. Detection of faults in low load condition was accomplished by Hilbert modulus with FFT (0.2\% slip) \cite{Puche-Panadero2009}, Hilbert modulus with ESPRIT (0.33\% slip) \cite{Xu2013}, Fourier analysis (1.38\% slip) \cite{dias2014}, Teager - Kaiser energy operator (0.4\% slip) \cite{pineda2013}. Fourier analysis of stator current envelop was carried out to detect BRB under a low slip of 0.11\% \cite{Sapena-Bano2016_low}. However, a high initial sampling of 50 kHz makes it disadvantageous for low-cost hardware implementation. 

The majority of the research has focused on detecting single and multiple BRB. Detection of partially BRB was demonstrated in \cite{Climente2015, didier2006, kia2005}. However, detecting partial BRB in low-slip applications, especially for the inverter-fed motor, is yet to be addressed. This thesis has detected a partially broken bar for the inverter-fed SCIM with a 0.2\% slip. Fault detection and classification literature have been predominantly based on traditional ML, such as neural networks, SVM, random forests, etc. However, the last five years have seen phenomenal growth in the application of deep learning (DL) for fault diagnosis and prognosis \cite{ han2019adaptive, zhao2019deep, lei2020applications}. Transfer learning \cite{shao2018highly} has been used to build upon existing networks for accelerating the training process. Razavi et al. \cite{razavi2017integrated} have used oversampling to enhance the class imbalanced vibration signal training dataset for machine learning. However, the requirement of large datasets encompassing the overall operating condition of a motor is quite challenging. Moreover, obtaining fault-data from a running plant for training is infeasible due to periodic maintenance. It is still an open problem for the DL models to work when trained using a different motor.

On the other hand, hypothesis testing using the generalized likelihood ratio test (GLRT) \cite{elbouchikhi2017motor, trachi2016novel} can only detect the presence or absence of the fault-component. The presence of an inherent, non-zero, and low-amplitude fault component in a healthy motor thus result in false alarms. A non-zero magnitude incorporated in the null hypothesis of the GLRT can reduce the false alarms, but it may result in a doubly non-central F-distributed test statistic. We propose a simple test based on the minimum distance receiver (\cite{kay1998fundamentals}, Pg. 112) to avoid complex distributions. The test can account for any fault-like magnitude that might be present in the healthy-motor data. Moreover, unlike DL-based methods, the test requires only a few healthy data cases to determine a threshold.

From the survey, we can conclude that spectral estimation of motor parameters with well-defined physics-based fault signatures is the most reliable diagnostic tool. There is a variety of available spectral estimators with their set of pros and cons. A review of the recent literature on spectral estimation is provided next.
 
\subsection{Recent Developments in Spectral Estimation Technologies}
Spectral estimation of a wide-sense stationary (WSS) signal buried in white noise has seen significant development for the last 50 years. The use of spectral estimates has been phenomenal in detecting SCIM faults \cite{samanta2018fast}. Lower computational complexity and the emergence of digital signal processors made Fourier transform indispensable for many ground-breaking applications. However, the low mean-squared error (MSE) requirement and improved resolution have driven the next level of research. MLE of the spectrum is asymptotically efficient. However, MLE is computationally complex and requires solving a non-convex multi-modal cost function \cite{trinh2018partial}. The orthogonality of eigenvectors of the Hermitian autocorrelation matrix has been of significant consequence. Beginning with the initial proposal by Pisarenko and later modifications by Schmidt has led to the MUSIC \cite{schmidt1986multiple} algorithm. MUSIC still has been able to encourage the research community for further development in iMUSIC \cite{tenneti2019imusic}, Gold-MUSIC \cite{rangarao2013gold}, etc. When the frequency components are closely-spaced, the interference due to the dominant components degrades the weak component's estimation. \cite{trinh2018partial} reduced the effect of interference mathematically by relegating the interfering eigenvectors to an arbitrary matrix before computing the MUSIC null-spectra. 

The subspace-based spectral estimators have inherently higher resolution, but they also require an accurate estimate of the model-order \cite{morency2018joint}. On the other hand, parametric model-based spectral estimators use the underlying model information to improve the resolution and decrease the MSE. Resolution of the model-based methods is dependent on the model-order instead of the data length ($N$). Although advantageous under data length restrictions, an arbitrary high-order selection can result in spurious peaks and spectral-splitting. The model-based methods require an inversion of the autocorrelation matrix. Faster implementation of the inversion can be achieved by using the structure of the Toeplitz matrix. Spectral estimation with structured-Toeplitz constraints can be found in \cite{hansen2018superfast}. The absence of regularization in AR-spectrum estimation leads to the inversion of an ill-conditioned matrix. The methods discussed till now have not considered any additional information about the estimated parameters. In \cite{kitagawa1985smoothness}, the authors have used a Gaussian prior for spectral estimation. The maximum-a-posterior (MAP) estimation solves a regularized least square problem. The regularized least square involves the inversion of a well-conditioned matrix. 

Djuric and Li \cite{djuric1995bayesian} provided the systematic use of prior distribution of signal and noise parameters for posterior-frequency estimation. \cite{fu2007map} used Tikhonov prior for both the phase and frequency parameter for MAP estimation. The use of a sparse Bayesian model with $l1$-norm minimization can be found in \cite{zhao2015robust}. A Bernoulli-Gaussian prior was used in \cite{badiu2017variational}. The discrete Bernoulli's distribution ascertains the model order, or the number of components present in the signal, whereas the Gaussian prior helps in localizing the frequency. The mixture model was then solved using the variational-Bayes framework.  Use of Bayesian inference for estimating spectrogram of a quasi-stationary signal was demonstrated in \cite{das2018dynamic}. The literature discussed until now have used a prior on the parameters of the harmonic model. However, the requirement of sequential detection of non-stationarity has been instrumental in defining the model-parameters as a Gauss-Markov first-order random walk \cite{chu2017new}. In \cite{chu2017new}, the authors have considered a linear system's output for a given input to estimate the time-varying parameters. In this work, we have modeled the input signal as a TVAR process. The parameters of the model are considered to be random, defined by a first-order random walk. 

Till now, we have considered the constant frequency under stationary conditions. However, most natural phenomenons are characterized by the presence of time-dependent frequency components. The next section examines the instantaneous frequency estimation developments.  

\subsection{Instantaneous Frequency Estimation and Signal Conditioning}
The extraction of frequency-based features has been the highlight of SCIM fault detection. However, the spectral fault features are numerically suppressed due to the presence of multiple high-magnitude frequency components. In such cases, signal conditioning is mainly accomplished by using notch filters \cite{ayhan2008use}. However, non-adaptive selection of central frequency and pass-band of the notch filters are disadvantageous when the motors operate under variable frequency and low-load conditions. Teager-Kaiser energy operator \cite{pineda2013application}, and oblique projection \cite{elbouchikhi2017motor}, on the other hand, have been mainly used for removing single components. We propose a method that can estimate, track, and eliminate multiple high-magnitude components to condition the input signal and extract the fault features appropriately. The proposed method is adaptive to variable frequency conditions and can remove targeted dominant components without affecting the closely-spaced fault components under low-load situations.

Frequency estimation techniques can be classified into stationary and non-stationary estimators depending on the statistical properties of the input signal. We can further classify stationary frequency estimators \cite{kay1988modern} into (a) classical methods like discrete Fourier transforms, power spectral densities; (b) high-resolution subspace-based spectral estimators like multiple signal classification (MUSIC) \cite{schmidt1986multiple}, estimation of signal parameters via rotational invariance technique (ESPRIT) \cite{roy1989esprit}; and (c) model-based approaches like autoregressive (AR), autoregressive moving average (ARMA), and moving average spectral estimators \cite{kay1988modern}. However, the majority of signals encountered in various applications are non-stationary \cite{hlawatsch2013time}, which can stem from the presence of time-varying frequencies. The use of instantaneous frequency (IF) estimation is well documented for various applications like speech processing \cite{kortlang2016auditory}, pulmonary disease diagnosis \cite{lozano2016automatic}, fault diagnosis of rotating machinery \cite{wang2018matching}, power quality assessment \cite{Chen2010}, etc. 

Non-stationary estimators for a long time have evolved around the linear transformations like short-time Fourier transform (STFT), wavelet transforms, and their quadratic counterparts - the spectrogram, and scalogram, respectively \cite{auger2013time}. Wigner-Ville-type distributions gained popularity by overcoming the difficulties of estimation bias and low-resolution of spectrograms. However, the cross-term problem of multiple components affected the frequency estimation invariably \cite{popovic2019efficient}. The length of the window used for the decomposition determines the resolving power of the estimator. The use of a Gaussian window-based STFT, commonly known as the S-transform, is quite popular. The variance of the Gaussian function decides the window length and hence, the resolution of the S-transform. An angle parameter was added to the Gaussian function to tune the window for variable chirp-rates in fractional Fourier transform \cite{tao2010short}. A single global parameter cannot select the optimum window for the transformation for multiple components with different chirp-rates. Hence, a local optimization was proposed to find the optimal window at each point in the time-frequency (TF) plane \cite{khan2013instantaneous}.

A new approach towards TF-analysis emerged with the inclusion of the phase information \cite{auger2013time}. The phase information concentrated the TF-energy and resulted in the development of the reassignment method (RM) \cite{auger1995improving} and the synchrosqueezed technique (SST) \cite{thakur2013synchrosqueezing}. The advantage of SST in extracting the modes of a non-stationary signal like empirical mode decomposition (EMD) makes it advantageous over RM \cite{daubechies2011synchrosqueezed}. For multicomponent signals with non-harmonic modes, Pham and Meignen extended the second-order SST \cite{oberlin2015second} to higher orders in \cite{pham2017high} for narrowband energy distribution and increased accuracy. The authors in \cite{pham2017high} have used the example of gravitational waves as a practical application. An iterative demodulation-based post-processing technique for improving the resolution of SST and RM under low signal to noise ratio (SNR) was proposed in \cite{wang2014matching}. 

The demodulation enhances the energy localization in the TF plane. For multiple components, the TF plane is partitioned to accommodate each component separately. The demodulation is repeated until a desired mean squared error (MSE) is achieved. A similar approach can also be found in \cite{yang2015component}, where multiple components are extracted individually by parametric demodulation using the estimate of the IF followed by bandpass filtering. This demodulation results in a stationary component having the initial frequency as its sole constituent. Therefore, the problem with frequency components that are closely-spaced or having intersecting trajectories needs to be addressed. In \cite{jensen2017fast}, a fast non-linear least square method was presented for finding the initial frequency and the chirp-rate for a linear-chirp model with real sinusoids. However, its performance in estimating quadratic frequency components needs validation. 

The Presence of outliers can perturb the least-squares estimate. A similar effect can be observed when we estimate IF from TF energy distribution. Random sample consensus (RANSAC) is generally used to reduce the impact of the outliers. Djurovic has used the RANSAC algorithm on Wigner distribution for single component estimation in \cite{djurovic2016wd}. The best noise performance was found to be -20 dB for 5 dB SNR. For the algorithm proposed in this thesis, the normalized-MSE (NMSE) with constant chirp-rate and 5 dB SNR is about -30 dB. Further improvement with RANSAC for multiple overlapping components can be found in \cite{djurovic2018qml}. 

The methods discussed till now are offline, and the IF estimation was carried out individually. Sequential estimation with minimum-MSE estimate of the time-varying AR (TVAR) parameters with Kalman filters and subsequent derivation of the power spectrum was reported in \cite{Aboy2005, khan2007expectation}. Both the observation and the state propagation models are linear in \cite{Aboy2005, khan2007expectation}. However, direct frequency estimation from the TVAR process leads to a non-linear observation model. As a result, the extended Kalman filter (EKF) \cite{Routray2002} and its variants are used to estimate the time-varying frequency of a single dominant component \cite{Routray2002} and its harmonics \cite{Chen2010}. The use of EKF in high noise environments with Fourier coefficients can be found in \cite{LaScala1996}. 

The authors in \cite{ahn2019bayesian} have discussed different strategies for state estimation of the Markovian jump system. Probabilistic neural networks for model selection and Kalman filter for optimal state estimation is reported in \cite{shi2018switched}. However, for IF estimation, the model is non-linear and may change at each instance of time. As a result, the number of test hypotheses grows exponentially with time, and the estimation becomes intractable \cite{ahn2019bayesian}. An adaptive notch filter formulation was employed using a recursive prediction error algorithm in \cite{Tichavsky1995}. Based on \cite{Tichavsky1995}, a self-optimizing method was developed in \cite{Niedzwiecki2006a} to estimate multiple frequencies, and further development was reported in \cite{Niedzwiecki2011}. The use of a phased locked loop for frequency estimation \cite{Fedele2014} is promising as it can estimate non-sinusoidal components. However, it is limited by its capability to estimate a single component. Based on the brief survey of the literature, we can summarize that
\begin{enumerate}
	\renewcommand{\theenumi}{\roman{enumi}}
	\item  The conventional methods of multi-component instantaneous frequency estimation require the signal to be decomposed into individual components before estimating the IF by finding the maximum energy over the TF plane \cite{khan2013instantaneous}.
	
	\item Decomposition of closely-spaced components and components having crossover frequencies are not yet solved.
	
	\item The Gabor-Heisenberg uncertainty principle limits the estimation performance in terms of its time and frequency resolutions where one can only be gained at the expense of the other \cite{vetterli2014foundations}. The uncertainty severely deteriorates the conventional methods' performance under higher chirp-rates and components exhibiting abrupt changes in frequency.
	
	\item The resolution of Fourier-based methods is dependent on the window length, and that of the model-based techniques are dependent on the number of parameters or the lags. Over-estimation of the model-order improves the resolution, but it can result in spurious ridges in the TF-plane. On the other hand, the sequential estimators mainly dealt with the estimation of a single component.
\end{enumerate}
%
% the conventional methods of multi-component frequency estimation require the signal to be decomposed into individual components before estimating the IF by finding the maximum energy over the TF plane \cite{khan2013instantaneous}. However, decomposition of closely-spaced components and components having crossover frequencies are not yet solved. Moreover, the Gabor-Heisenberg uncertainty principle limits the estimation performance in terms of its time and frequency resolutions where one can only be gained at the expense of the other \cite{vetterli2014foundations}. The uncertainty severely deteriorates the performance of the conventional methods under higher chirp-rates and components exhibiting abrupt changes in frequency. The resolution of Fourier based methods are dependent on the window length, and that of the model-based techniques are dependent on the number of parameters or the lags. Over-estimation of the model order improves the resolution, but it can result in spurious ridges in the TF-plane. On the other hand, the sequential estimators mainly dealt with the estimation of a single component.